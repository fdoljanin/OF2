\documentclass{article}

\usepackage[T1]{fontenc}
\usepackage[utf8]{inputenc}
\usepackage{lmodern}
\newcommand{\cc}{\text{\v{c}}}
\let\oldref\ref
\renewcommand{\ref}[1]{(\oldref{#1})}

\usepackage{fancyhdr}
\usepackage{amsmath}
\usepackage{amsfonts}
\usepackage{graphicx} 

\begin{document}
\section*{Sila nabijene šipke na naboj}
Vektorski zapis diferencijala sile iznosi:
\begin{align}
  d\vec{F_q} & =k\frac{q \cdot dQ}{r_{12}^2}\hat{r}_{21}         \\
             & =k\frac{q \cdot \lambda dL}{r_{12}^2}\hat{r}_{21} \\
  \label{eq:TotalDifferential}
             & =k\frac{q\cdot \lambda dL}{x^2+L^2}\hat{r}_{21}
\end{align}
Diferencijal po vektorskim komponentama iznosi:
\begin{align}
  dF_x & = dF\frac{x}{\sqrt{x^2+L^2}} \\
  dF_y & = dF\frac{L}{\sqrt{x^2+L^2}}
\end{align}
Iznos komponente u odnosu na $|\vec{dF_q}|$ \ref{eq:TotalDifferential} proizlazi iz geometrije zadatka:
\begin{align}
  dF_x & = dF\frac{x}{\sqrt{x^2+L^2}}                            \\
       & = k\frac{q \lambda dL}{x^2+L^2}\frac{x}{\sqrt{x^2+L^2}} \\
       & = kq{\lambda}x\frac{dL}{\sqrt{(x^2+L^2)^3}}
\end{align}
Ukupna sila $F_x$ iznosi, gdje \ref{Eq:InsertResolvedX} proizlazi iz integrala \ref{Eq:ExpandedIntegralX}:
\begin{align}
  F_x & = \int dF_x                                                          \\
      & = \int_{0}^{a} kq{\lambda}x\frac{dL}{\sqrt{(x^2+L^2)^3}}             \\
      & = kq{\lambda}x\int_{0}^{a}\frac{dL}{\sqrt{(x^2+L^2)^3}}              \\
  \label{Eq:InsertResolvedX}
      & = kq{\lambda}x\left[\frac{1}{x^2}\frac{L}{\sqrt{L^2+x^2}}\right]_0^a \\
      & = kq{\lambda}x\frac{1}{x^2}\left(\frac{a}{\sqrt{a^2+x^2}} - 0\right) \\
      & = kq\frac{{\lambda}a}{x\sqrt{a^2+x^2}} = \frac{kQq}{x\sqrt{a^2+x^2}}
\end{align}
Prikazani integral možemo riješiti $\tan$ substitucijom:
\begin{align}
  \label{Eq:ExpandedIntegralX}
  \int\frac{dL}{\sqrt{(x^2+L^2)^3}}
   & = \frac{1}{x^3}\int\frac{dL}{\left(1+\frac{L^2}{x^2}\right)^\frac{3}{2}}                                 \\
   & = \begin{vmatrix}
         \tan\theta = \frac{L}{x} \\
         \frac{1}{\cos^2\theta}d\theta = \frac{dL}{x}
       \end{vmatrix}                                                            \\
   & = \frac{1}{x^3}\int\frac{x}{\cos^2\theta}\frac{d\theta}{\left(1+\tan^2\theta\right)^{\frac{3}{2}}}       \\
   & = \frac{x}{x^3}\int\frac{1}{\cos^2\theta}\frac{d\theta}{\left(\frac{1}{\cos^2\theta}\right)^\frac{3}{2}} \\
   & = \frac{1}{x^2}\int\cos\theta{d}\theta                                                                   \\
   & = \frac{1}{x^2}\sin\theta + C                                                                            \\
   & = \frac{1}{x^2}\sin\tan^{-1}\left(\frac{L}{x}\right) + C                                                 \\
   & = \frac{1}{x^2}\frac{L}{\sqrt{L^2+x^2}} + C
\end{align}
Ukupna sila $F_y$ iznosi, gdje \ref{Eq:InsertResolvedY} proizlazi iz integrala \ref{Eq:ExpandedIntegralY}:

\begin{align}
  F_y & = {\int}dF\frac{L}{\sqrt{x^2+L^2}}                                     \\
      & = \int_{0}^{a}k\frac{q{\lambda}dL}{x^2+L^2}\frac{L}{\sqrt{x^2+L^2}}    \\
      & = kq{\lambda}\int_{0}^{a}\frac{L dL}{\left(x^2+L^2\right)^\frac{3}{2}} \\
  \label{Eq:InsertResolvedY}
      & = kq{\lambda}\left[-\frac{1}{\sqrt{x^2+L^2}}\right]_{0}^{a}            \\
      & = \frac{kqQ}{a}\left(\frac{1}{x} - \frac{1}{\sqrt{x^2+a^2}}\right)
\end{align}
Prikazani integral možemo riješiti $u$-substitucijom:
\begin{align}
  \int\frac{L dL}{\left(x^2+L^2\right)^\frac{3}{2}}
  \label{Eq:ExpandedIntegralY}
   & = \begin{vmatrix}
         u = x^2+L^2 \\
         du = 2LdL
       \end{vmatrix}                                   \\
   & = \frac{1}{2}\int\frac{du}{u^\frac{3}{2}}          \\
   & = -\frac{1}{2\cdot\frac{1}{2}}u^{-\frac{1}{2}} + C \\
   & = -\frac{1}{\sqrt{x^2+L^2}} + C
\end{align}
\end{document}