\documentclass{article}

\usepackage[T1]{fontenc}
\usepackage[utf8]{inputenc}
\usepackage{lmodern}
\newcommand{\cc}{\text{\v{c}}}
\let\oldref\ref
\renewcommand{\ref}[1]{(\oldref{#1})}

\usepackage{fancyhdr}
\usepackage{amsmath}
\usepackage{amsfonts}
\usepackage{graphicx} 

\begin{document}
\section*{Frekvencija maksimalne snage}
U pronalasku frekvencije maksimalne snage koristit ćemo se općepoznatim izrazom za trenutnu snagu na otporniku $P = I^2R$. Kako je riječ o izmjeničnoj struji, razmatramo efektivnu vrijednost struje $I_{ef}$:
\begin{equation}
  \label{eq:EffectivePower}
  P = I_{ef}^2R = \left(\frac{I_{max}}{\sqrt{2}}\right)^2R = \frac{I_{max}^2R}{2}
\end{equation}
Odnos struje i napona možemo zapisati kao:
\begin{equation}
  \label{eq:CurrentVoltageRelation}
  I = \frac{U}{Z} = \frac{U_{max}\sin(\omega t)}{\sqrt{R^2 + (\omega L - \frac{1}{\omega C})^2}}
\end{equation}
$I_{max}$ se događa periodično, kada je $\sin(\omega t) = 1$, dakle ovisno o varijabli $t$.\\Tada iz \ref{eq:EffectivePower} i \ref{eq:CurrentVoltageRelation} za $P$ imamo:
\begin{align}
  P & = \frac{I_{max}^2R}{2}                                                                                   \\
    & = \frac{\left(U_{max}R\right)^2}{2\left(R^2 + (\omega L - \frac{1}{\omega C})^2\right)}                  \\
  \label{eq:EffectivePowerTotal}
    & = \frac{\left(U_{max}R\right)^2}{2} \cdot \frac{1}{\left(R^2 + (\omega L - \frac{1}{\omega C})^2\right)}
\end{align}
Maksimalna snaga u odnosu na varijablu $\omega$ dobija se deriviranjem:
\begin{align}
  \frac{dP}{d\omega} & = \frac{\left(U_{max}R\right)^2}{2} \cdot \frac{d}{d\omega}\left(\frac{1}{\left(R^2 + (\omega L - \frac{1}{\omega C})^2\right)}\right)                                                          \\
                     & = \frac{\left(U_{max}R\right)^2}{2} \cdot \frac{-1}{\left(R^2 + (\omega L - \frac{1}{\omega C})^2\right)^2} \cdot \frac{d}{d\omega}\left(R^2 + (\omega L - \frac{1}{\omega C})^2\right)         \\
                     & = \frac{\left(U_{max}R\right)^2}{2} \cdot \frac{-1}{\left(R^2 + (\omega L - \frac{1}{\omega C})^2\right)^2} \cdot \left(2(\omega L - \frac{1}{\omega C})\cdot (L + \frac{1}{\omega^2C}) \right) \\
                     & = 0
\end{align}
Očigledno je da $\omega L - \frac{1}{\omega C} = 0$, pa je maksimalna snaga ostvarena kada
\begin{equation}
  \omega_0 = \frac{1}{\sqrt{LC}} \approx 224\text{Hz}
\end{equation}
i uvrštavanjem u \ref{eq:EffectivePowerTotal} dobije se $P_{max} = 500\text{W}$.
\newpage
\section*{Frekvencija upola manje snage}
Frekvenciju upola manje snage dobit ćemo uvrštavanjem u formulu \ref{eq:EffectivePowerTotal} za snagu:
\begin{align}
  P_{1}                                                                                                  & = \frac{1}{2}P_{max}                                                      \\
  \frac{\left(U_{max}R\right)^2}{2} \cdot \frac{1}{\left(R^2 + (\omega L - \frac{1}{\omega C})^2\right)} & = \frac{1}{2} \cdot \frac{\left(U_{max}R\right)^2}{2} \cdot \frac{1}{R^2} \\
  2R^2                                                                                                   & = R^2 + \left(\omega L - \frac{1}{\omega C}\right)^2                      \\
  \label{eq:HalfPowerSolution}
  L\omega^2 \pm R\omega - \frac{1}{C}                                                                    & = 0
\end{align}
Rješenje kvadratne jednažbe \ref{eq:HalfPowerSolution} iznosi:
\begin{equation}
  \omega_{1,2} = \frac{\pm R \pm \sqrt{R^2 + 4L/C}}{2L}
\end{equation}
Uzimajući u obzir da samo pozitivna rješenja imaju fizikalnog smisla, uvrštavanjem brojeva dobije se: $\omega_1 \approx 221\text{Hz}$ i $\omega_2 \approx 226\text{Hz}$.
\end{document}